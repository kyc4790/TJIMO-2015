\documentclass[11pt]{article}
\usepackage[paperwidth=8.5in, paperheight=11in]{geometry}
\usepackage{algorithm}
\usepackage{algorithmicx}
\usepackage[noend]{algpseudocode}

\usepackage{../tjimo}

\newcommand{\sevenpoints}{If you have any questions at any time during the day, call Samuel at 703-568-2167.}
\newcommand{\righthead}{\fdbox{Important}{Coaching Packet}}

\begin{document}

\begin{small}
\section{Schedule}
\begin{tabular}{l l l }
  8:00 & Lecture Hall Prep & \\
  9:30 & Team Assignments & take students to assigned room \\
  & 9:40 & Individual Concepts Review \\
  & 10:10 & Practice Power Round* \\
  & 10:40 & Practice Power Review* \\
  & 10:55 & Practice Team Round* \\
  & 11:25 & Practice Team Review* \\
  & 11:40 & Practice Guts Round* \\
  & 11:50 & Practice Guts Review* \\
 12:00 & Lunch (provided) \\
  1:00 & Official Competition \\
  3:45 & Snacks in Cafeteria \\
  4:00 & Awards Ceremony \\
  4:30 & Cleanup \\
\end{tabular}

\vspace{0.5cm}

\noindent You must arrive at 8:00 a.m. Note that we did not say ``8:01'' or ``8:00 and 1 second'', but ``8:00''. 
Report to the Lecture Hall, where you may leave your belongings for the day. The Lecture Hall is just beyond the Cafeteria and participant registration.
While the students are taking the individual round from 8:45 to 9:30, you will be in the Lecture Hall reviewing all your packets to make sure that you
are completely comfortable with the materials and getting your team assignment. This will be the last opportunity you have to clear up any issues you
may have with the IMO. At 9:30 you will go down to the cafeteria to get your team of seven or eight bright students who think you are absolutely amazing.
Take them to your assigned room. Do not disturb anything in the room except for desks and the black/white boards (be sure you use the right types
of markers on whiteboards). Return everything to its original state after the competition. You personally are responsible for the condition of your team's room.
Once your team is set up, introduce yourselves and play an icebreaker activity if you wish. Have your team members make name tags so they can get to know each other.
Make absolute certain that your team members are nice and encouraging to each other. Last year, a participant left early because he was bullied by his teammates.
This behavior must not be tolerated.

\noindent *These are ``suggested'' starting times.  Depending on the needs of your team, spend more time reviewing concepts/rounds in the areas you feel appropriate.

\section{Review}
Suggested Time for Review: 30 Minutes \\
\noindent The first important activity is the Individual and Concepts Review.  First, ask them if they have any specific questions about the Individual round; they should still have their problem sheet with them.  They will DEFINITELY not have solved every problem, and many students may be shy about asking questions.  Pick some difficult problems and ask them if they answered it, and how they solved it if they did.  Have some of them come up to the board and explain their solutions if they seem to grasp the concepts clearly.  Your AM Packet should contain answers and explanations to all individual and practice problems so you can step up when needed.  \\
\indent Make sure to get the kids as involved as possible as soon as you can, and explain things that they don't all understand.  Both of these are vital to your team's success and the success of the IMO as a whole.  Figure out what they don't understand, and pick a topic to go into a lot of detail and explain.  Make sure not to go overboard and leave enough time for each of the practice rounds.

\section{Practice Power Round}
\noindent Time: 30 Minutes \\
\noindent Suggested Time for Review: 15 Minutes\\
\noindent The first round you should practice is the Power Round. The Practice Power Round will take 30 minutes; the real one will take 45 minutes. These questions center around a single topic; in this case, Modular Arithmetic.  Each student has a copy of the round (on white paper).  They must submit their team solution on the answer sheets. Unfortunately, dividing up the problems tends to lead to the eighth deadly sin, an action contrary to the entire spirit of the IMO and so despicable it is to be avoided at all costs. We speak, of course, of plunging. Plunging is when the smartest kid on the team says, ``Oh! I see how to do parts 1, 2, and 3. Done! Now I'll look at part 4!'' While it may earn your team a few points, you'll all lose out in the end. The rest of the team will become lost, confused, disheartened, eventually leading to a revolt, in which they will tie your shoelaces together, cast you out of the room, and lock the door behind you (This has happened before). Once the students are done with the practice power round, look at what they've written. If they don¡¯t give enough detail, or they make an assumption without explaining it, or they violate some other rule, tell them so that they can fix it later. That's why it's a practice round. Give your students about 30 minutes to work on this, but you might want to stop them before then if it seems they are getting stuck so you can spend more time explaining.

\section{Practice Team Round}
\noindent Time: 30 Minutes \\
\noindent Suggested Time for Review: 15 Minutes\\
\noindent The Practice Team Round has 10 problems and should take 30 minutes. The real Team Round in the afternoon also has 10 problems to be solved in 30 minutes.  There are a few tips for you to give the students if you want to make them think you know what you're doing. Suggest the divide-and-conquer approach. The best results come from working in pairs. It's also a good idea to have all the answers checked, especially by people who didn't originally work on the problem.

\section{Practice Guts Round}
\noindent Time: 10 Minutes \\
\noindent Suggested Time for Review: 10 Minutes\\
\noindent The Practice Guts Round has 12 problems to be done in 10 minutes; the real one has 36 problems to be done in 30 minutes.  This is the most interesting round. Teams will work through distinct sets of four problems each. They must turn in whatever answers they have for set $n$ before they receive set $n+1$, so you may want to strategize with them about this new dimension. The problems are arranged roughly in increasing order of difficulty. The problems are weighted equally in terms of point value. Your team will need to decide how to balance speed and accuracy, since answers, once turned in, are final.  During the afternoon round, you will be responsible for obtaining the problem sets for your team. This means walking, however briskly, to the problem distributor, who will exchange problem set $n + 1$ for your team's answers to problem set $n$, and then returning in similar fashion with the new problems. Be sure your team's number is on each answer sheet.

\section{General Practice Ideas}
\noindent And finally, we include a word on the practice rounds in general: ``Practice.'' That's what they're there for. Not to decide who's better than who, within a team or between teams, and not to antagonize anybody. In the morning, you should try to get your team having as much fun as is safely possible. Between each round, explain what they did right or wrong, and advise them on better strategy. If you have any good ideas from your own experience, share them. It is recommended, though, that you let each of the rounds run its course without interruption because that's how things will have to be during the afternoon round.

\section{Lunch}
\noindent Runners will come around during the practice rounds to make sure everything is running smoothly. The lunch starts at noon. Please be prompt, but do not come early. At some point during lunch, you'll also need to pick up the afternoon packet which contains all the problems and answer sheets for the rest of the day. Check that it has all of the necessary materials (you'll have a list) and then put it somewhere safe from your students. They can leave all their stuff in the room during lunch. After lunch you will collect them and bring them back to your room for ...

\section{The Official Competition}
\noindent Congratulations! You've made it to the actual competition. These are the three hours that we've
spent the last month preparing for. The real competitions are just like the practice ones except they are, in fact, the official ones, so your only job is to pass out and collect papers and make sure that no un-ethical, un-friendly, un-mathematical, or un-sportsmanlike conduct takes place. Do not leave your room while the official rounds are happening. After each round there will be a five minute break for your team to catch its breath, and, if necessary, run to the restrooms. There is no time for socializing, especially for you!  All rounds must start on time. We will make announcements over the loudspeaker to coordinate everyone. When the end of a round is called, collect your team's sheet(s) immediately. Make sure your team's number is on the answer sheet. At the announced end of each round, there will be runners in the halls waiting to receive your team's answer sheets. You must hand the correct sheets to the runners at this time or else your team will not get scored, and you know what that means... Once these interesting rounds are done, clean up your room and bring your team to the cafeteria for the actual Guts and Orienteering Rounds. The Guts Round, which is explained earlier in this packet, will then commence. The Orienteering Round will follow afterwards. \\

\noindent \textbf{Official Rounds (Listen for Loudspeaker): }
\begin{enumerate}
  \item Power Round (45 min)
  \item Team Round (30 min, 10 problems)
  \item Guts Round (In Cafeteria; 30 min, 36 problems)
\end{enumerate}

\section{The Orienteering Round}
\noindent Your team members have been sitting around all day, and by now their muscles have probably atrophied. To remedy that, we make them walk, not run, back and forth across the second floor to get each of eight problems in turn. Any students caught running will result in their team being disqualified. Problem 0 is picked up in the cafeteria. The solution of each problem is the number of the room in which their next problem will be located; it is different for each team and problem, though the problems are the same from team to team. Once (if) they get the last problem, they need to find their way back to the cafeteria. The whole team, including one coach (you) moves as one unit from room to room. However, you are not allowed to give any directions in this round.  Your job is simply to watch them and keep things under some semblance of control.  In their eagerness to get the next problem, your team will act in ways that might seem slightly uncivilized to a well-mannered mathlete: running, shouting, screaming, and deserting other team members. All of those, particularly the last, are specifically forbidden. In order to get their next problem, a coach (you) must be with the team and must confirm that all six team members are present. Room numbers are displayed as numbers from 1-30. Look for the sign posted next to each room labeling each room from 1-30.

\section{The End}
\noindent After the Orienteering Round, we will hold the awards ceremony, and then the students will disperse. At 5:00 (after cleaning up, of course), you will disperse. Please plan to have a ride at 5:00. In order to collect your nine service points, you will need to work for all nine hours. Thank you in advance for your help, and we'll see you on Saturday.

\end{small}	
\end{document}

